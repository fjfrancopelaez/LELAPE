\chapter{How to use LELAPE}
\section{Formatting input data}
%
You are supposed to have performed experiments on some memory element. Tests were performed as:
\begin{itemize}
	\item \textbf{Static}: The device was written, sent to standby mode, irradiated and eventually read. The content after the irradiation was compared to the initially writing.
	\item \textbf{Pseudostatic}: Similar to static ones, but standby intervals are shorter than the irradiation time and the memory is read several times during the irradiation. Usually, flipped bits are corrected on the fly.
\end{itemize}

In both cases, the researcher saves information about the bitfilps: Word address, read content, 
In order to use LELAPE, radiation test data must be converted to a matrix with three or four columns. The meaning of the columns is the following:
\begin{itemize}
	\item First Column: Word Address where bitflips were observed.
	\item Second Column: Content in the word address after the radiation tests.
	\item Third Column: Content in the word address before the irradiation.
	\item Fourth Column: In pseudostatic tests, cycle in which the bitflip was observed. This column can be omitted in the static tests or replaced by a column full of ones. 
\end{itemize} 

LELAPE needs this elements to be converted to a UInt32 matrix\footnote{In Julia, typing variable is optional but encouraged to speed up calculations.} so it is important that all the elements of the matrix, including the cycle label, are in this format or, at least, in some kind of integer. This makes dangerous labelling the fourth column with words or letters. 

A simple solution consists in grouping the data results in CSV format and read this text file with \texttt{readdlm}, included in the \text{DelimitedFiles} package. In the Jupyter folder, you can see some examples that can guide you to adapt your own data. One advantage of this function is that it can automatically convert the data to the required format, as shown in the Jupyter notebooks.

\section{Setting up the analysis}
%
Before starting the analysis, we must define some additional variables to indicate the software how to proceed. These variables are the followong:
%
\begin{itemize}
	\item \textbf{LA}: Variable in integer format. It indicates the memory size in words (not in bits!). It is often a power of 2.
	\item \textbf{WordWidth}: Also an integer, it indicates the number of bits per word. Typically 8, 16, 32 but other values are possible.
	\item \textbf{Operation}: A string variable to indicate the mathematical operation used to create the DV set. So far, only two options are implemented:
	\begin{itemize}
		\item \textit{XOR}: Addresses are xored bit to bit. This mode is set with the \texttt{``XOR''} value.
		\item \textit{Positive subtraction}: The absolute value of the difference of addresses is returned. It is marked with \texttt{"POS"}. 
	\end{itemize}
	In practice, we have observed that the former is appropriate for SRAMs and the latter for FPGAs. However, this idea may be erroneous due to the use of few and partial radiation test data.
	%
	\item \textbf{UsePseudoAddress}: The pseudoaddress is defined as follows: let us suppose that we have observed a bitflip in the $k$-th position of the $NWA$-th word address, The word width is $W$ and $k=0$ corresponds to the least significant bit, $W-1$ to the most significant one. Hence, the pseudoaddress of the bitflip is:
	\[
		PSA = NWA\cdot W + k
	\]
	This value is full of meaning in FPGAs since just returns the position of the cell in the bitflips. It is completly artificial in some SRAMs but somehow analysis using the pseudoaddress instead of the word address are more accurate and efficient. 
	
	The researcher can set this variable to \texttt{true} or \texttt{false} at will.
	%
	\item \textbf{KeepCycles}: In pseudostatic tests, this boolean variable indicates that the system must use the information about the cycles (fourth column) or just using the set of data as a whole. 
	%
	\item \textbf{TraceRuleLength}: This an integer variable with 1,2 or 3 as allowed values. LELAPE looks for anomalously repeated elements in the DV set with very few ones in binary representation. The user can decide if looks for elements with 1, 2 or 3 ones or less and include them as candidates to detect pairs. 
	%
	\item \textbf{$\varepsilon$}: A float number always positive but close to 0. If the expected number of elements repeated $k$ times in the DV set is lower than $\varepsilon$, we must consider this number of repetitions impossible. If higher, we determine than at least an element can appear $k$ times just due to randomness. Default value is 0.05.
	
	A very low value of $\varepsilon$ will exclude false positive but also genuine values relating pair of addresses in an MCU. On the contrary, if it is chosen too low, false positives might be taken as good ones.
	%
	\item \textbf{LargestMCUSize}: During the search of critical DV values, LELAPE starts to group addresses in provisional MCUs that grow large and large as new possible critical DV values are tested. Unfortunately, sometimes this process does not find a stable solution and goes on looking for it despite being unrealistic. This parameter is used to stop the calculation since it informs the software of not considering MCUs with more than \textit{LargestMCUSize} addresses. By default, it is set to 200.
\end{itemize}

 \section{Functions in the module}
 
 \subsection*{ConvertToPseudoADD}
 \begin{itemize}
 	\item \textbf{Arguments}: 
 		\begin{itemize}
 			\item \textit{Method 1: }\textbf{DATA}::Array\{UInt32\}, \textbf{WordWidth}::Int
 			\item \textit{Method 2: }\textbf{DATA}::Array\{UInt32\}, \textbf{WordWidth}::Int, \textbf{KeepCycle}::Bool
 		\end{itemize}
 	\item   \textbf{Output}: Array\{UInt32, 2\}, or Matrix\{UInt32\}.
 	\item  This function looks for the flipped bits between words in the same row but in the second
 	 and third columns of the DATA matrix. It does not matter if there are several bitflips, since they are independently coundted. 
 	 The pseudoaddress  of each bitflip, defined as \[\text{WORDADDRESS}\times\text{Wordwith}+\text{Bitposition}\] is returned as the first column of the output.
 	
 	 If there is information about the different cycles, it can be kept in the optional second column in the output with the condition of declaring \textbf{KeepCycle} true. If cycle information is absent, the second column is filled with 1s.
 	
 \end{itemize}

\subsection*{ExtractFlippedBits}
\begin {itemize}
	\item \textbf{Arguments}:
		\begin{itemize}
			\item \textit{Method 1: }\textbf{WORD}::UInt32, \textbf{PATTERN}::UInt32, \textbf{Wordwidth}::Int
			\item \textit{Method 2: }\textbf{WORD}::UInt16, \textbf{PATTERN}::UInt16, \textbf{Wordwidth}::Int
			\item \textit{Method 3: }\textbf{WORD}::UInt8, \textbf{PATTERN}::UInt8, \textbf{Wordwidth}::Int
		\end{itemize}
	\item \textbf{Output}: :Array\{Int,1\}, or Vector\{Int\}
	\item This function allows discovering the position of different bits between WORD
	 and PATTERN. It also verifies that both values are coherent with the \textbf{Wordwidth},
	 meaning that neither of them are higher than \(2^{Wordwidth}-1\). A vector, never larger than \textbf{Wordwidth} is returned. If \textbf{WORD} and \textbf{PATTERN} are equal, the output is a void vector.
\end{itemize}

\subsection*{CheckMBUs}

TBD

\subsection*{DetectAnomalies\_SelfConsis}
\begin{itemize}
	\item \textbf{Arguments}:
	\begin{itemize}
		\item  \textit{Method 1: }\textbf{DATA}::Array\{UInt32, 2\}, 
		\textbf{WordWidth}::Int,
		\textbf{LN0}::Int,
		\textbf{Operation}::String,
		\textbf{UsePseudoADD}::Bool,
		\textbf{KeepCycle}::Bool,
		\textbf{\(\varepsilon\)}::AbstractFloat,
		\textbf{LargestMCUSize}::Int
		%
		\item  \textit{Method 2: }\textbf{DATA}::Array\{UInt32, 2\}, 
		\textbf{WordWidth}::Int,
		\textbf{LN0}::Int,
		\textbf{Operation}::String,
		\textbf{UsePseudoADD}::Bool,
		\textbf{KeepCycle}::Bool,
		\textbf{\(\varepsilon\)}::AbstractFloat
		%
		\item  \textit{Method 3: }\textbf{DATA}::Array{UInt32, 2}, 
		\textbf{WordWidth}::Int,
		\textbf{LN0}::Int,
		\textbf{Operation}::String,
		\textbf{UsePseudoADD}::Bool,
		\textbf{KeepCycle}::Bool
	\end{itemize}
	%
	\item \textbf{Output: } Array\{UInt32, 2\}	
	%
	\item This function will calculate the anomalies in the set of addresses using the SelfConsistency principle.

	 First of all, let us know the inputs:
	 \begin{itemize}
	
		\item   \textbf{DATA}: A matrix with 3 or 4 columns. 
			 \begin{itemize}
			 	\item The first one contains the word addresses in UInt32 format.
			 	\item The second one shows the content read in the memory after the irradiation.
			 	\item The third one, the pattern that should be inside.
			 	\item  The fourth one is optional and shows the namber of the read cycle if the   memory was read and corrected several times during the irradiation.
			 \end{itemize}
		\item   \textbf{WordWidth}: The size of each word in bits, usually 8, 16. 32, etc.
		\item   \textbf{LN0}: The memory size in words (not in bits!!!). In many cases, \(2^N\).
		\item   \textbf{Operation}: A string variable to indicate the preferred operation to calculate
	    the DVSET. Only two operations are allowed: 
	    \begin{itemize}
	    	\item "XOR": XORing bit to bit.
	    	\item "POS": abs(a-b)
	    \end{itemize}
		\item  \textbf{UsePseudoADD}: A boolean variable. It allows to indicate that the user wants to user word addresses (false). If true, a pseudoaddress is assigned to each bit and calculated
	    as \(\text{WORADDRESS}\times\text{WordWidth} + \text{BitPosition}\). Full of sense in FPGA since it is just the position   of the bit in the bitstream, it has not physical interpretation in memories BUT works!!!!
		\item   \textbf{KeepCycle}: If true, the function looks for the fourth column and uses it to calculate the DVSet.
		\item   \textbf{\(\varepsilon\)}: A small positive integer number to determine the threshold that defines when a number of repetitions are impossible to occur. Set by default to 0.05 if not provided among the input arguments.
		\item  \textbf{LargestMCUSize}: This value indicates the largest possible size for MCUs. It has not physical sense  and is only used to stop the program if unreallistic events occur. Set to 200 if not given as an input.
	
	   The funcion return an \(N\times 2\) UInt32 matrix. The first column contains the anomalously repeated values of the DV SET compatible with the SelfConsistency test. The second one contains the number of times they appear in the DV set. Due to format integrity reasons, this column is expressed in unnatural UInt32 format. 
	  
	  It is advisable a latter conversion into Int to make this column more readable.
	\end{itemize}
\end{itemize}
%
\subsection*{DetectAnomalies\_Shuffle\_Rule}
 TBD
 %
 \subsection*{DetectAnomalies\_Trace\_Rule}
 %
 TBD
 %
 \subsection*{DetectAnomalies\_MCU\_Rule}
 %
 TBD
 %
 \subsection*{DetectAnomalies\_FullCheck}
 %
 TBD
 %
 \subsection*{MCU\_Indexes}
 %
 \begin{itemize}
 	\item \textbf{Arguments}: 
 		\begin{itemize}
 			\item \textit{Method 1: } \textbf{DATA}::Matrix\{UInt32\}, 
 			\textbf{OPERATION}::String,
 			\textbf{Markers}:: Vector\{UInt32\}, 
 			\textbf{UsePseudoADD}::Bool, 
 			\textbf{WordWidth}::Int,
 			\textbf{LimitMCUSize}:: Int
 			%
 			\item \textit{Method 2: } \textbf{DATA}::Matrix\{UInt32\}, 
 			\textbf{OPERATION}::String,
 			\textbf{Markers}:: Vector\{UInt32\}, 
 			\textbf{UsePseudoADD}::Bool, 
 			\textbf{WordWidth}::Int
 			%
 			\item \textit{Method 3: } \textbf{DATA}:Matrix\{UInt32\}, 
 			\textbf{OPERATION}::String,
 			\textbf{Markers}:: Vector\{UInt32\}, 
 			\textbf{UsePseudoADD}::Bool
 			%
 			\item \textit{Method 4: } \textbf{DATA}::Matrix\{UInt32\}, 
 			\textbf{OPERATION}::String,
 			\textbf{Markers}:: Vector\{UInt32\}, 
 		\end{itemize}
 	\item \textbf{Output}: Matrix\{Int\}
 	%
 	\item This functions uses the \textbf{DATA} set to look for pairs of addresses which treated with 	 \textbf{OPERATION} yield one of the \textbf{MARKERS}. If an \textbf{ADDRESS} is related to other two addresses, 	 a 3--bit MCU appears (and so on.) The rest of parameters are used to provide necessary  information to use the \textbf{PSEUDOADDRESS} instead of the \textbf{WORD ADDRESS}. 
	 
	 More information about the inputs:
	 
	 \begin{itemize}
	 	\item DATA: A matrix with 3 or 4 columns. 
	 		\begin{itemize}
	 			\item  The first one contains the word addresses in UInt32 formata.
	 			\item        The second one shows the content read in the memory after the irradiation.
	 			\item        The third one, the pattern that should be inside.
	 			\item        The fourth one is optional and shows the namber of the read cycle if the  memory was read and corrected several times during the irradiation.
	 		\end{itemize}
	 	%       
	 	\item \textbf{OPERATION}: A string variable to indicate the preferred operation to calculate
	 	the DVSET. Only two operations are allowed: 
	 	\begin{itemize}
	 		\item "XOR": XORing bit to bit.
	 		\item "POS": abs(a-b)
	 	\end{itemize}
	 	%
	 	\item \textbf{UsePseudoADD}: A boolean variable. It allows to indicate that the user wants to user
	 	word addresses (false). If true, a pseudoaddress is assigned to each bit and calculated
	 	as \[\text{WORDADDRESS}\times\text{WordWidth} + \text{BitPosition.}\] Full of sense in FPGA since it is just the position
	 	of the bit in the bitstream, it has not physical interpretation in memories BUT works!!!!
	 	%
	 	\item \textbf{WordWidth}: The size of each word in bits, usually 8, 16. 32, etc.
	 	%
	 	\item \textbf{LargestMCUSize}: This value indicates the largest possible size for MCUs. It has not physical sense
	 	and is only used to stop the program if unreallistic events occur. Initially set to 200.
	 \end{itemize}
  
	 Concerning the OUTPUT: It provides an integer NMCU\(\times\)LMCU matrix, NMCU being the number of detected MCUs and
	 LMCU the size of the largest reconstructed MCU. Every value different than 0 must be determined as follows:
	 \begin{enumerate}
	 	\item UsePseudoADD = false: The index indicates the row in DATA with the address in the MCU.
	 	\item UsePseudoADD = true: It provides the index in the PSEUDOADDRESS derived SET. If the exact position of
	 	the bitcell is required, DATA should be treated with ConvertToPseudoADD() and
	 	the index used in the resulting matrix.
	 \end{enumerate}
 
     In both cases, if the size of the MCU is smaller than LMCU, the row will be filled with zeros until
     reaching the desired length. For example, if the content of a row is [5 7 9 0 0], it must be interpreted
     as a 3-bit MCU involving addresses indexed with 5, 7 \& 9 in an experiment in which at least a 5-bit 
     MCU (and nothing larger) was observed.
     
 	Finally, if the index of an address does not appear in the returned matrix, it should be interpreted as  isolated and belonging to an SBU.
 	
 \end{itemize}
%
\subsection*{Classify\_Addresses\_in\_MCU}
%
\begin{itemize}
	\item \textbf{Arguments}: 
		\begin{itemize}
			\item \textit{Method 1}: \textbf{DATA}:: Matrix\{UInt32\}, 
			\textbf{Indexes}:: Matrix\{Int\}, 
			\textbf{UsePseudoADD}:: Bool, 
			\textbf{WordWidth}:: Int
			%
			\item \textit{Method 2}: \textbf{DATA}:: Matrix\{UInt32\}, 
			\textbf{Indexes}:: Matrix\{Int\}, 
			\textbf{UsePseudoADD}:: Bool 
			%
			\item \textit{Method 3}: \textbf{DATA}:: Matrix\{UInt32\}, 
			\textbf{Indexes}:: Matrix\{Int\}
		\end{itemize}
	%
	\item \textbf{Output}: Vector:: \{Any\}
	%
	\item The purpose of this function is to classify the addresses (or pseudoaddresses) with bitflips transform the matrix of INDEXES got from MCU\_Indexes() into a Vector of matrices, called SOLUTION, which is eventually returned as OUTPUT.
	 
	The length of SOLUTION is the size of the largest observed MCU(s), NLMCU. Thus, SOLUTION[1] is 	a N x NLMCU matrix in which every row contains the addresses or pseudoaddresses of the NLMCU 	bitflips involved in this MCU. N is the number of observdd NLMCU-bit MCUs.
	
	SOLUTION[2] is devoted to events with M = NLMCU-1 bits. As before, it is a matrix with NLMCU-1 	rows and an undetermined number of rows.
	
	Finally, SOLUTION[NLMCU] is just a simple vector with the addresses not involved in MCUs. Obviously, these are the SBUs.
\end{itemize}

\subsection*{TheoAbundance\_XOR}
%
\begin{itemize}
	\item \textbf{Arguments:}
	%
		\begin{itemize}
			\item \textit{Method 1}: \textbf{NR}: :Int, \textbf{NB}::Int, \textbf{LN}::Int, \textbf{UsingDV}::Bool
			\item \textit{Method 2}: \textbf{NR}: :Int, \textbf{NB}::Int, \textbf{LN}::Int
		\end{itemize}
	\item \textbf{Output:} AbstractFloat
	%
	\item This function allows calculating the expected number of values repeated 	 \textbf{NR} times after \textbf{NB} bitflips in a memory with \textbf{LA} words with \textbf{W} bits per word
	 with XOR operation.
	
	 \textbf{NR} must be an integer number higher or equal than 0; 
	 \textbf{NB} is an integer number supposed to be higher than 1.
	 \textbf{LN} is an integer number, and indicates the size of the elements in the set
	 where elements are chosen.
	
	 Sometimes, the user provides directly the DV set so an additional boolean
	 input is provided to take into account this fact. It is usually disabled. In this
	 case, NB PLAYS THE ROLE OF NDV.
	 
	 Equations were got from Eq.12 of the Appendix.C in 
	 F. J. Franco et al., "\textit{Statistical Deviations From the Theoretical Only-SBU
	 Model to Estimate MCU Rates in SRAMs,}" in IEEE Transactions on Nuclear
	 Science, vol. 64, no. 8, pp. 2152-2160, Aug. 2017,
	 doi: 10.1109/TNS.2017.2726938. 
	
	 Lawfully avalaible for free on https://eprints.ucm.es/id/eprint/43874/
\end{itemize}
%
\subsection*{TheoAbundance\_POS}
%
\begin{itemize}
	\item \textbf{Arguments}:
		%
	\begin{itemize}
		\item \textit{Method 1}: \textbf{NR}: :Int, \textbf{NB}::Int, \textbf{LN}::Int, \textbf{UsingDV}::Bool
		\item \textit{Method 2}: \textbf{NR}: :Int, \textbf{NB}::Int, \textbf{LN}::Int
	\end{itemize}
	\item \textbf{Output:} AbstractFloat
	%
	\item This function allows calculating the expected number of values repeated 	 NR times after NB bitflip in a memory with LA words with W bits per word   	supposing to have used the POSITIVE subtraction.
 
  \textbf{NR} must be an integer number higher or equeal than 0
  \textbf{NB} is an integer number supposed to be higher than 1.
  \textbf{LN} is an integer number, and indicates the size of the elements in the set
  where elements are chosen.
 
  Sometimes, the user provides directly the DV set so an additional boolean
  input is provided to take into account this fact. It is usually disabled. In this
  case, NB PLAYS THE ROLE OF NDV.
 
  Equations were got from Eq.2 of the Appendix in J. C. Fabero et al., "\textit{Single Event Upsets Under 14-MeV Neutrons in a 28-nm
 SRAM-Based FPGA in Static Mode,}" in IEEE Transactions on Nuclear Science, vol.
  67, no. 7, pp. 1461-1469, July 2020, doi: 10.1109/TNS.2020.2977874.
  
  Lawfully avalaible for free download on \href{https://eprints.ucm.es/id/eprint/59496/}{https://eprints.ucm.es/id/eprint/59496/}
\end{itemize}
%
\subsection*{TheoAbundance}
%
	\begin{itemize}
	\item \textbf{Arguments}:
		\begin{itemize}
			\item \textit{Method 1:} \textbf{NR}::Int, \textbf{NB}::Int, \textbf{LN}::Int, \textbf{Operation}:: String, \textbf{UsingDV}::Bool
			\item \textit{Method 1:} \textbf{NR}::Int, \textbf{NB}::Int, \textbf{LN}::Int, \textbf{Operation}:: String
		\end{itemize}
	\item \textbf{Output}: AbstractFloat
	\item This is an Alias for TheoAbundance\_POS() or TheoAbundance\_XOR().
	
	 \textbf{NR} must be an integer number higher or equeal than 0
	 \textbf{NB} is an integer number supposed to be higher than 1.
	 \textbf{LN} is an integer number, and indicates the size of the elements in the set
	 where elements are chosen.
	 \textbf{Operation} is "XOR" or "POS". 
	 
	   Sometimes, the user provides directly the DV set so an additional boolean
	 input is provided to take into account this fact. It is usually disabled. In this
	 case, NB PLAYS THE ROLE OF NDV. If \textbf{UsingDV} is not provided, it is supposed to be false.
\end{itemize}
%
\subsection*{MaxExpectedRepetitions}
%
\begin{itemize}
	\item \textbf{Arguments}: 
		\begin{itemize}
			\item \textit{Method 1}: \textbf{NDV}::Int, \textbf{LN}::Int, \textbf{Operation}:: String, \(\varepsilon\):: AbstractFloat	
			\item \textit{Method 2}: \textbf{NDV}::Int, \textbf{LN}::Int, \textbf{Operation}::String
		\end{itemize}
	\item \textbf{Output}: Int
	\item The purpose of this funcion is to determina the maximum number of expected
	repetitions. in a DV set taken from a memory with size equal to \textbf{LN}.
	In general, it is the first integer such that its theoretical abundance is
	lower than \(\varepsilon\). If this threshold is not provided, it is assumed to be 0.01.
\end{itemize}
%
\subsection*{CorrectNBitFlips}
%
	\begin{itemize}
		\item \textbf{Arguments}: \textbf{NBF}::Int, \textbf{LN}::Int
		\item \textbf{Output}: Float64
		\item This function tries to correct the number of bitflips to compensate cells hit twice that
		escape from inspection. 
		%
		\begin{itemize}
			\item \textbf{NBF}: Number of bitflips. Theoretically, SBUs but they are impossible to be distinguished 
			from other kinds of bitflips. Therefore, a simple approach is taken.
			\item \textbf{LN}: Memory size in BITS!!!!!
		\end{itemize}	 
		%
		 Expression is taken from Eq. 6 in F. J. Franco, J. A. Clemente, H. Mecha and R. Velazco, 
		 "\textit{Influence of Randomness During the Interpretation of Results From Single-Event Experiments 
		 on SRAMs}," IEEE Transactions on Device and Materials Reliability, vol. 19, no. 1, pp. 104-111, 
		 March 2019, doi: 10.1109/TDMR.2018.
		 2886358.		
	\end{itemize}
%
\subsection*{NF2BitMCUs}
%
\begin{itemize}
	\item \textbf{Arguments}: 
		\begin{itemize}
			\item \textit{Method 1}: \textbf{NSBU}::Int, \textbf{LA}::Int, \textbf{METHOD}::String, \textbf{D}::Int, \textbf{WordWidth}::Int, \textbf{UsePseudoAddress}::Bool
			\item \textit{Method 2}: \textbf{NSBU}::Int, \textbf{LA}::Int, \textbf{METHOD}::String, \textbf{D}::Int, \textbf{WordWidth}::Int
		\end{itemize}
	\item \textbf{Output}: Float64
	\item It indicates the expected number of false 2-bit MCUs that will occur 
	 in a memory with \textbf{LA} words with \textbf{WORDWIDTH} bits each in which \textbf{NSBU} SBUs have occurred. In this analysis,
	
	 MCUS are sought using some grouping method (\textbf{METHOD}) with a generalized distance \textbf{D}.
	
	 Admitted values for \textbf{METHOD} and \textbf{D} are the following:
	 
	 \begin{enumerate}
	 	\item METHOD: "MBU" --> Only MBUs are sought. In this case, D is the Wordwidth.
	 	\item METHOD: "MHD" --> Only possible if the user has been able to place the ExtractFlippedBits 	 	cell in the XY plane. Two cells are related if |x1-x2|+|y1-y2| <= D. This 
	 	is the Manhattan distance.
	 	\item METHOD: "IND" --> Only possible if the user has been able to place the ExtractFlippedBits
	 	cell in the XY mplane. Two cells are related if max(|x1-x2|,|y1-y2|) <= D.
	 	In mathematics, this is the "infinite distance".
	 	\item METHOD: "THD" --> Only valid if pairs of bitflips are located in a linear bitstream and if the
	 	distace between cells is smaller than D: |x1-x2| <= D.
	 	\item METHOD: "XOR" --> Related pairs are got by means of statistical deviations. Addresses are XORed 
	 	and only if the value is one of the D possible critical values. If the WORD Addresses  
	 	is used instead of PSEUDOADDRESS, the memory size must be expressed in WORDs, 
	 	LA = LN/WordWidth.
	 	IF SO, THE WORDWIDTH MUST BE PROVIDED.
	 	\item METHOD: "POS" --> Identical to the previous one but with positive subtraction instead of XOR.
	 \end{enumerate}
	 
	   Everything can be found in Eq. 11 of F. J. Franco, J. A. Clemente, G. Korkian, 
	   J. C. Fabero, H. Mecha and R. Velazco, "\textit{Inherent Uncertainty in the Determination of Multiple 
	   Event Cross Sections in Radiation Tests,}" IEEE Transactions on Nuclear Science, vol. 67, no. 7, 
	   pp. 1547-1554, July 2020, doi: 10.1109/TNS.2020.2977698.
\end{itemize}
%
\subsection*{NF3BitMCUs}
%
\begin{itemize}
	\item \textbf{Arguments}: 
	\begin{itemize}
		\item \textit{Method 1}: \textbf{NSBU}::Int, \textbf{NMU2}::Int, \textbf{LN}::Int, \textbf{METHOD}::String, \textbf{D}::Int, WordWidth::Int
		\item \textit{Method 2}: \textbf{NSBU}::Int, \textbf{NMU2}::Int, \textbf{LN}::Int, \textbf{METHOD}::String, \textbf{D}::Int
	\end{itemize}
	\item \textbf{Output}: Tuple{Float64, Float64}
	\item     It indicates the expected number of false 3-bit MCUs that will occur 
	 in a memory with LN bits in which NSBU SBUs and NMU2 2-bit MCUs have occurred. 
	 In this analysis, MCUS are sought using some grouping --method (METHOD) with a generalized 
	 distance D.
	
	MCUS are sought using some grouping method (\textbf{METHOD}) with a generalized distance \textbf{D}.
	
	Admitted values for \textbf{METHOD} and \textbf{D} are the following:
	
	\begin{enumerate}
		\item METHOD: "MBU" --> Only MBUs are sought. In this case, D is the Wordwidth.
		\item METHOD: "MHD" --> Only possible if the user has been able to place the ExtractFlippedBits 	 	cell in the XY plane. Two cells are related if |x1-x2|+|y1-y2| <= D. This 
		is the Manhattan distance.
		\item METHOD: "IND" --> Only possible if the user has been able to place the ExtractFlippedBits
		cell in the XY mplane. Two cells are related if max(|x1-x2|,|y1-y2|) <= D.
		In mathematics, this is the "infinite distance".
		\item METHOD: "THD" --> Only valid if pairs of bitflips are located in a linear bitstream and if the
		distace between cells is smaller than D: |x1-x2| <= D.
		\item METHOD: "XOR" --> Related pairs are got by means of statistical deviations. Addresses are XORed 
		and only if the value is one of the D possible critical values. If the WORD Addresses  
		is used instead of PSEUDOADDRESS, the memory size must be expressed in WORDs, 
		LA = LN/WordWidth.
		IF SO, THE WORDWIDTH MUST BE PROVIDED.
		\item METHOD: "POS" --> Identical to the previous one but with positive subtraction instead of XOR.
	\end{enumerate}
	
	Everything can be found in Eq. 11 of F. J. Franco, J. A. Clemente, G. Korkian, 
	J. C. Fabero, H. Mecha and R. Velazco, "\textit{Inherent Uncertainty in the Determination of Multiple 
		Event Cross Sections in Radiation Tests,}" IEEE Transactions on Nuclear Science, vol. 67, no. 7, 
	pp. 1547-1554, July 2020, doi: 10.1109/TNS.2020.2977698.
	
	In this paper, it was demonstrated that it is mathematically impossible to get an exact value. Therefore,  optimistic and pessimistic results are provided.
\end{itemize}
%
\subsection*{NPairs}
\begin{itemize}
	\item \textbf{Arguments}:
		\begin{itemize}
			\item \textit{Method 1}: \textbf{DATA}::Array{UInt32}
			\item \textit{Method 2}: \textbf{DATA}::Array{UInt32}, \textbf{UsePseudoAdd}::Bool
			\item \textit{Method 3}: \textbf{DATA}:: Array{UInt32}, \textbf{UsePseudoAdd}::Bool, \textbf{WordWidth}:: Int
			\item \textit{Method 4}: \textbf{DATA}::Array{UInt32}, \textbf{UsePseudoAdd}:: Bool, \textbf{WordWidth}:: Int, \textbf{KeepCycle}:: Bool
			\item \textit{Method 5}: \textbf{N}::Int
		\end{itemize}
	\item \textbf{Output}: Int
	%
	\item \textbf{DATA} is a 3 or 4-column matrix derived from the loaded CSV file and each row containing the word address (\#1), the read word after the tests (\#2), the initial pattern (\#3) and the number of reading cycle when the error was observed. If this last column is not providedJ or \textbf{KeepCycle} is false, the system works as if only one cycle was done. 
	
	The function provides the number of pairs of addresses taken during each cycle regarding predictions of the Only-SBU model. For example, let us suppose that we have done 2 cycles, observing in the first one 30 events, and 40 in the second. The number of possible pairs is the addition of the pairs in each cycle:
	\[
		\frac{1}{2}\cdot 30\cdot(30-1)+\frac{1}{2}\cdot 40\cdot(40-1) = 1215
	\]
	Thus, 1215 pairs can be formed. If we had not taken into account the existence of cycles, the number of pairs woudl have been\[	\frac{1}{2}\cdot (30+40)\cdot(30+40-1) = 2415\]
	however, many of them are unreal since were taken in different times!
	
	If \textbf{UsePseudoAdd} is set to true, the system looks for the position of the bitflips inside the word and uses the pseudoaddress (\(\text{WordAddress}\times\text{WordWidth}+\text{Position}\)). Thus, it is necessary to provide the \textbf{WordWidth} value (8, 16, 32, ...). If this value is not provided. Default values for this variables are \textbf{UsePseudoAdd}~=~true, \textbf{WordWidth}~=~1, \textbf{KeepCycle}~=~1. Using \textbf{DATA} as the only argument is appropriate to analyze values directly in pseudoaddress format taken during one only cycle. This is the case, for example, of FPGAs configuration memory. 
	
	There is a final method, just saying the number of observed pairs, \textbf{N}. In this case, the function returns \textbf{N}\(\cdot\)(\textbf{N}-1)/2.
	
\end{itemize}
%
\subsection*{NTriplets}
\begin{itemize}
	\item \textbf{Arguments}:
	\begin{itemize}
		\item \textit{Method 1}: \textbf{DATA}::Array{UInt32}
		\item \textit{Method 2}: \textbf{DATA}::Array{UInt32}, \textbf{UsePseudoAdd}::Bool
		\item \textit{Method 3}: \textbf{DATA}:: Array{UInt32}, \textbf{UsePseudoAdd}::Bool, \textbf{WordWidth}:: Int
		\item \textit{Method 4}: \textbf{DATA}::Array{UInt32}, \textbf{UsePseudoAdd}:: Bool, \textbf{WordWidth}:: Int, \textbf{KeepCycle}:: Bool
		\item \textit{Method 5}: \textbf{N}::Int
	\end{itemize}
	\item \textbf{Output}: Int
	%
	\item Similar to \textit{Npairs(...)}, but calculating the expected number of triplets instead of pairs.
\end{itemize}